\chapter{ Технологический раздел}
\label{cha:impl}
Данные реализации разрабатывались на языке C, использовался компилятор gcc-8.2.1.
Замер времени производится с помощью библиотеки \textit{time.h}. Данная библиотека позволяет производить замер тиков, откуда можно получать время в секундах. 

\section{ Листинг}
Алгоритм Левенштейна: 
\begin{lstlisting}[style=CStyle]
int levensteinDistance( char* s1, char* s2,
                        int ( * matrix)[MAX_STRING_SIZE][MAX_STRING_SIZE] )
{
    size_t len1 = strlen(s1);
    size_t len2 = strlen(s2);

    matrixInit(matrix, len1, len2);

    int cost = 1;
    for(int i = 1; i <= len1; i++)
        for(int j = 1; j <= len2; j++) {
            if(s1[i-1] == s2[j-1])
                cost = 0;
            else
                cost = 1;
            ( * matrix)[i][j] = min( 3,
                        ( * matrix)[i-1][j]+1,
                        ( * matrix)[i][j-1]+1,
                        ( * matrix)[i-1][j-1]+cost );
        }
}
\end{lstlisting}

Листинг алгоритма Дамерау -- Левенштейна:
\begin{lstlisting}[style=CStyle]
int levensteinDistance( char* s1, char* s2,
                        int ( * matrix)[MAX_STRING_SIZE][MAX_STRING_SIZE] )
{
    size_t len1 = strlen(s1);
    size_t len2 = strlen(s2);

    matrixInit(matrix, len1, len2);

    int cost = 1;
    for(int i = 1; i <= len1; i++)
        for(int j = 1; j <= len2; j++) {
            int flag = 0;
            if(s1[i-1] == s2[j-1])
                cost = 0;
            else
                cost = 1;
            int l1= min( 3,
                        ( * matrix)[i-1][j]+1,
                        ( * matrix)[i][j-1]+1,
                        ( * matrix)[i-1][j-1]+cost );
            int l2 = l1;
            if(i>=2 && j>=2)
                if(s1[i-1] == s2[j-2] && s2[j-1] == s1[i-2])
                    l2 = ( * matrix)[i-2][j-2]+1;
            ( * matrix)[i][j] = min(2, l1, l2);
        }
}
\end{lstlisting}

Листинг алгоримта Левенштейна в рекурсивной реализации:
\begin{lstlisting}[style=CStyle]
int levensteinDistance( char* s1, int i, char* s2, int j )
{
    if(i == 0) return j;
    if(j == 0) return i;

    return min( 3,levensteinDistance(s1, i-1, s2, j) + 1,
                  levensteinDistance(s1, i, s2, j-1) + 1,
                  levensteinDistance(s1, i-1, s2, j-1) + match(s1[i], s2[j]) );
}
\end{lstlisting}

Листинг функции match():
\begin{lstlisting}[style=CStyle]
int match(char c, char d)
{
    if(c == d) return 0;
    else return 1;
}
\end{lstlisting}
