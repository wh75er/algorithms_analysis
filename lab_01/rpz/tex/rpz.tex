%% Преамбула TeX-файла

% 1. Стиль и язык
\documentclass[utf8, 12pt]{G7-32} % Стиль (по умолчанию будет 14pt)

% Остальные стандартные настройки убраны в preamble.inc.tex.
\sloppy

% Настройки стиля ГОСТ 7-32
% Для начала определяем, хотим мы или нет, чтобы рисунки и таблицы нумеровались в пределах раздела, или нам нужна сквозная нумерация.
\EqInChapter % формулы будут нумероваться в пределах раздела
\TableInChapter % таблицы будут нумероваться в пределах раздела
\PicInChapter % рисунки будут нумероваться в пределах раздела
\usepackage{slashbox}

% Добавляем гипертекстовое оглавление в PDF
\usepackage[
bookmarks=true, colorlinks=true, unicode=true,
urlcolor=black,linkcolor=black, anchorcolor=black,
citecolor=black, menucolor=black, filecolor=black,
]{hyperref}

% Изменение начертания шрифта --- после чего выглядит таймсоподобно.
% apt-get install scalable-cyrfonts-tex

\IfFileExists{cyrtimes.sty}
    {
        \usepackage{cyrtimespatched}
    }
    {
        % А если Times нету, то будет CM...
    }

\usepackage{graphicx}   % Пакет для включения рисунков

% С такими оно полями оно работает по-умолчанию:
% \RequirePackage[left=20mm,right=10mm,top=20mm,bottom=20mm,headsep=0pt]{geometry}
% Если вас тошнит от поля в 10мм --- увеличивайте до 20-ти, ну и про переплёт не забывайте:
\geometry{right=20mm}
\geometry{left=30mm}


% Пакет Tikz
\usepackage{tikz}
\usetikzlibrary{arrows,positioning,shadows}

% Произвольная нумерация списков.
\usepackage{enumerate}

% ячейки в несколько строчек
\usepackage{multirow}

% itemize внутри tabular
\usepackage{paralist,array}

% Центрирование подписей к плавающим окружениям
\usepackage[justification=centering]{caption}

% объявляем новую команду для переноса строки внутри ячейки таблицы
\newcommand{\specialcell}[2][c]{%
	\begin{tabular}[#1]{@{}c@{}}#2\end{tabular}}



% Настройки листингов.
\ifPDFTeX
% Листинги

\usepackage{listings}
\usepackage{wrapfig}
% Значения по умолчанию
\lstset{
  basicstyle= \footnotesize,
  breakatwhitespace=true,% разрыв строк только на whitespacce
  breaklines=true,       % переносить длинные строки
%   captionpos=b,          % подписи снизу -- вроде не надо
  inputencoding=koi8-r,
  numbers=left,          % нумерация слева
  numberstyle=\footnotesize,
  showspaces=false,      % показывать пробелы подчеркиваниями -- идиотизм 70-х годов
  showstringspaces=false,
  showtabs=false,        % и табы тоже
  stepnumber=1,
  tabsize=4,              % кому нужны табы по 8 символов?
  frame=single,
  escapeinside={(*}{*)}, %выделение
  literate={а}{{\selectfont\char224}}1
  {б}{{\selectfont\char225}}1
  {в}{{\selectfont\char226}}1
  {г}{{\selectfont\char227}}1
  {д}{{\selectfont\char228}}1
  {е}{{\selectfont\char229}}1
  {ё}{{\"e}}1
  {ж}{{\selectfont\char230}}1
  {з}{{\selectfont\char231}}1
  {и}{{\selectfont\char232}}1
  {й}{{\selectfont\char233}}1
  {к}{{\selectfont\char234}}1
  {л}{{\selectfont\char235}}1
  {м}{{\selectfont\char236}}1
  {н}{{\selectfont\char237}}1
  {о}{{\selectfont\char238}}1
  {п}{{\selectfont\char239}}1
  {р}{{\selectfont\char240}}1
  {с}{{\selectfont\char241}}1
  {т}{{\selectfont\char242}}1
  {у}{{\selectfont\char243}}1
  {ф}{{\selectfont\char244}}1
  {х}{{\selectfont\char245}}1
  {ц}{{\selectfont\char246}}1
  {ч}{{\selectfont\char247}}1
  {ш}{{\selectfont\char248}}1
  {щ}{{\selectfont\char249}}1
  {ъ}{{\selectfont\char250}}1
  {ы}{{\selectfont\char251}}1
  {ь}{{\selectfont\char252}}1
  {э}{{\selectfont\char253}}1
  {ю}{{\selectfont\char254}}1
  {я}{{\selectfont\char255}}1
  {А}{{\selectfont\char192}}1
  {Б}{{\selectfont\char193}}1
  {В}{{\selectfont\char194}}1
  {Г}{{\selectfont\char195}}1
  {Д}{{\selectfont\char196}}1
  {Е}{{\selectfont\char197}}1
  {Ё}{{\"E}}1
  {Ж}{{\selectfont\char198}}1
  {З}{{\selectfont\char199}}1
  {И}{{\selectfont\char200}}1
  {Й}{{\selectfont\char201}}1
  {К}{{\selectfont\char202}}1
  {Л}{{\selectfont\char203}}1
  {М}{{\selectfont\char204}}1
  {Н}{{\selectfont\char205}}1
  {О}{{\selectfont\char206}}1
  {П}{{\selectfont\char207}}1
  {Р}{{\selectfont\char208}}1
  {С}{{\selectfont\char209}}1
  {Т}{{\selectfont\char210}}1
  {У}{{\selectfont\char211}}1
  {Ф}{{\selectfont\char212}}1
  {Х}{{\selectfont\char213}}1
  {Ц}{{\selectfont\char214}}1
  {Ч}{{\selectfont\char215}}1
  {Ш}{{\selectfont\char216}}1
  {Щ}{{\selectfont\char217}}1
  {Ъ}{{\selectfont\char218}}1
  {Ы}{{\selectfont\char219}}1
  {Ь}{{\selectfont\char220}}1
  {Э}{{\selectfont\char221}}1
  {Ю}{{\selectfont\char222}}1
  {Я}{{\selectfont\char223}}1
}

% Стиль для псевдокода: строчки обычно короткие, поэтому размер шрифта побольше
\lstdefinestyle{pseudocode}{
  basicstyle=\small,
  keywordstyle=\color{black}\bfseries\underbar,
  language=Pseudocode,
  numberstyle=\footnotesize,
  commentstyle=\footnotesize\it
}

% Стиль для обычного кода: маленький шрифт
\lstdefinestyle{realcode}{
  basicstyle=\scriptsize,
  numberstyle=\footnotesize
}

% Стиль для коротких кусков обычного кода: средний шрифт
\lstdefinestyle{simplecode}{
  basicstyle=\footnotesize,
  numberstyle=\footnotesize
}

% Стиль для BNF
\lstdefinestyle{grammar}{
  basicstyle=\footnotesize,
  numberstyle=\footnotesize,
  stringstyle=\bfseries\ttfamily,
  language=BNF
}

% Определим свой язык для написания псевдокодов на основе Python
\lstdefinelanguage[]{Pseudocode}[]{Python}{
  morekeywords={each,empty,wait,do},% ключевые слова добавлять сюда
  morecomment=[s]{\{}{\}},% комменты {а-ля Pascal} смотрятся нагляднее
  literate=% а сюда добавлять операторы, которые хотите отображать как мат. символы
    {->}{\ensuremath{$\rightarrow$}~}2%
    {<-}{\ensuremath{$\leftarrow$}~}2%
    {:=}{\ensuremath{$\leftarrow$}~}2%
    {<--}{\ensuremath{$\Longleftarrow$}~}2%
}[keywords,comments]

% Свой язык для задания грамматик в BNF
\lstdefinelanguage[]{BNF}[]{}{
  morekeywords={},
  morecomment=[s]{@}{@},
  morestring=[b]",%
  literate=%
    {->}{\ensuremath{$\rightarrow$}~}2%
    {*}{\ensuremath{$^*$}~}2%
    {+}{\ensuremath{$^+$}~}2%
    {|}{\ensuremath{$|$}~}2%
}[keywords,comments,strings]

% Подписи к листингам на русском языке.
\renewcommand\lstlistingname{\cyr\CYRL\cyri\cyrs\cyrt\cyri\cyrn\cyrg}
\renewcommand\lstlistlistingname{\cyr\CYRL\cyri\cyrs\cyrt\cyri\cyrn\cyrg\cyri}

\else
\usepackage{local-minted}
\fi

\usepackage{tabu}

% Полезные макросы листингов.
% Любимые команды
\newcommand{\Code}[1]{\textbf{#1}}


\begin{document}

\frontmatter % выключает нумерацию ВСЕГО; здесь начинаются ненумерованные главы: реферат, введение, глоссарий, сокращения и прочее.

% Команды \breakingbeforechapters и \nonbreakingbeforechapters
% управляют разрывом страницы перед главами.
% По-умолчанию страница разрывается.

% \nobreakingbeforechapters
% \breakingbeforechapters

% Также можно использовать \Referat, как в оригинале
%\begin{abstract}
%	Титульный лист. Эта страница нужна мне, чтобы не сбивалась нумерация страниц
%	\cite{Dh}
%	\cite{Bayer}
%	\cite{Habr1}
%	\cite{Noise_func}
%	\cite{Ulich}

%Это пример каркаса расчётно-пояснительной записки, желательный к использованию в РПЗ проекта по курсу РСОИ.

%Данный опус, как и более новые версии этого документа, можно взять по адресу (\url{https://github.com/rominf/latex-g7-32}).

%\end{abstract}
% НАЧАЛО ТИТУЛЬНОГО ЛИСТА
\begin{center}
	\hfill \break
	\textit{
		\normalsize{Государственное образовательное учреждение высшего профессионального образования}}\\ 
	
	\textit{
		\normalsize  {\bf  «Московский государственный технический университет}\\ 
		\normalsize  {\bf имени Н. Э. Баумана»}\\
		\normalsize  {\bf (МГТУ им. Н.Э. Баумана)}\\
	}
	\noindent\rule{\textwidth}{2pt}
	\hfill \break
	\noindent
	\makebox[0pt][l]{ФАКУЛЬТЕТ}%
	\makebox[\textwidth][c]{«Информатика и системы управления»}%
	\\
	\noindent
	\makebox[0pt][l]{КАФЕДРА}%
	\makebox[\textwidth][r]{«Программное обеспечение ЭВМ и информационные технологии»}%
	\\
	\hfill\break
	\hfill \break
	\hfill \break
	\hfill \break
	\normalsize{\bf Р А С Ч Ё Т Н О - П О Я С Н И Т Е Л Ь Н А Я\space\space З А П И С К А}\\
	\normalsize{\bf к лабораторной работе на тему:}\\
	\hfill \break
	\large{Рекурентные соотношения: расстояние между строками}\\
	\hfill \break
	\hfill \break
	\hfill \break
	\hfill \break
	\hfill \break	
	\normalsize {
		\noindent
		\makebox[0pt][l]{Студент}%
		\makebox[\textwidth][c]{}%
		\makebox[0pt][r]{{$\underset{\text{(Подипсь, дата)}}{\underline{\hspace{6cm}}}$ \space Киселев А.М.}}
	}\\
	\hfill \break	
	\normalsize {
		\noindent
		\makebox[0pt][l]{Преподаватель}%
		\makebox[\textwidth][c]{ ~~~~~~~~      }%
		\makebox[0pt][r]{{$\underset{\text{(Подпись, дата)}}{\underline{\hspace{6cm}}}$ \space Волкова Л.Л.}}
	}
	\hfill \break
	\hfill \break
	\hfill \break
	\hfill \break
\end{center}
\hfill \break
\hfill \break
\begin{center} Москва 2018\end{center}

\thispagestyle{empty} % 
% КОНЕЦ ТИТУЛЬНОГО ЛИСТА


%%% Local Variables: 
%%% mode: latex
%%% TeX-master: "rpz"
%%% End: 


\tableofcontents

%\include{10-defines}
%\include{11-abbrev}

\Introduction

Целью работы является изучение нахождения трудоемкости на материале алгоритма перемножения матриц Винограда. Для достижения поставленной цели необходимо решить следующие задачи:

\begin{itemize}
\item Изучить алгоритм Винограда;
\item Получить улучшенную версию Винограда и сравнить ее с обычной реализацией;
\item Провести оценку трудоемкости алгоритма;
\item Рассмотреть лучший и худший случаи при перемножении матриц данным алгоритмом;
\item Привести эксперементальное подтверждение различий во временной эффективности оптимизированного алгоритма и обычного при помощи разработанного ПО на материале замеров процессорного времени выполнения реализации на варьирующихся размерах перемножаемых матриц;
\item Описать и обосновать полученные результаты о выполненной работе;
\end{itemize}


\mainmatter % это включает нумерацию глав и секций в документе ниже

\chapter{ Аналитический раздел}
\label{cha:analysis}

\section{ Описание алгоритмов}
Пусть даны 2 квадратные матрицы размерностью $[l * m]$ и $[m * n]$ соответственно.

\subsection{ Классический алгоритм перемножения матриц.}
Операция умножения двух матриц выполнима только в то случае, если число столбцов в первом сомножителе равно числу строк во втором; в этом случае говорят, что матрицы согласованы. В частности, умножение всегда выполнимо, если оба сомножителя -- квадратные матрицы одного и того же порядка.

\subsection{ Алгоритм умножения матриц Винограда.}
Если посмотреть на результат умножения двух матриц, то видно, что каждый элемент в нем представляет собой скалярное произведение соответствующих строки и столбца исходных матриц. Можно заметить также, что такое умножение допускает предварительную обработку, позволяющую часть работы выполнить заранее. Рассмотрим два вектора $V = (v_1, v_2, v_3, v_4)$ и $W = (w_1, w_2, w_3, w_4)$. Их скалярное произведение равно $(V*W) = (v_1+w_2)*(v_2+w_1)+(v_3+w_4)*(v_4+w_3) - v_1*v_2 - v_3*v_2 - v_3*v_4-w_1*w_2-w_3*w_4$. Кажется, что второе выражение задает больше работы, чем первое: вместо четырех умножений мы насчитываем их шесть, а вместо трех сложений - десять. Менее очевидно, что выражение в правой части последнего равенства допускает предварительную обработку: его части можно вычислить заранее и запомнить для каждой строки первой матрицы и для каждого столбца второй. Таким образом, несмотря на то, что второе выражение требует вычисления большего количества операций, чем первое: вместо четырех умножений - шесть, а вместо трех сложений - десять, выражение в правой части последнего равенства допускает предварительную обработку: его части можно вычислить заранее и запомнить для каждой строки первой матрицы и для каждого столбца второй, что позволяет выполнять для каждого элемента лишь первые два умножения и последующие пять сложений, а также дополнительно для сложения.

\subsection{ Распараллеливание задач на CPU}
Процессор является одной из главных частей компьютера, которая позволяет выполнять планирование и вычисления. Мобильные системы (телефоны, планшеты, ноутбуки) и большинство десктопов имеют один процессор. Рабочие станции и сервера иногда могут похвастаться двумя или больше процессорами на одной материнской плате. Поддержка нескольких центральных процессоров в одной системе требует многочисленных изменений в её дизайне. Как минимум, необходимо обеспечить их физическое подключение (предусмотреть несколько сокетов на материнской плате), решить вопросы идентификации процессоров , согласования доступов к памяти и доставки прерываний (контроллер прерываний должен уметь маршрутизировать прерывания на несколько процессоров) и, конечно же, поддержки со стороны операционной системы.

Если процессоров несколько, то каждый из них имеет собственный разъём на плате. У каждого из них при этом имеются полные независимые копии всех ресурсов, таких как регистры, исполняющие устройства, кэши. Делят они общую память — RAM. Память может подключаться к ним различными и довольно нетривиальными способами, но это отдельная история, выходящая за рамки этой статьи. Важно то, что при любом раскладе для исполняемых программ должна создаваться иллюзия однородной общей памяти, доступной со всех входящих в систему процессоров.

Как уже упоминалось выше, работа с несколькими процессам требует серьезных изменений в дизайне системы, работающей с ними. Но стоимость несколько процессоров весьма не маленькая и, к тому же, если стоимость переключений между ними слишком велика, весь смысл от параллелизма пропадает. Тогда люди начали задумывать о том как расположить несколько процессов максимально близко друг к другу, желательно на одном кристале. На помощь здесь приходят ядра. Эти ядра во всем идентичны друг другу, но работают независимо.

Ядра могут иметь общие кэши и другие ресурсы, что заметно ускоряет работу, т.к. сокращает задержки обмена данных между ядрами, особенно если они работают надо общей задачей.

В один момент была предстаавлена новая технология компанией Intel -- HyperThreading, идея которой заключалась в выполнении одной программы разными потоками. Потоки так же могут дели Кэши и память. В отличие от «настоящих» ядер, являющихся полными и независимыми копиями, в случае HT в одном процессоре дублируется лишь часть внутренних узлов, в первую очередь отвечающих за хранение архитектурного состояния — регистры. Исполнительные же узлы, ответственные за организацию и обработку данных, остаются в единственном числе, и в любой момент времени используются максимум одним из потоков.

Пусть (x, y, z), где x — это число процессоров, y — число ядер в каждом процессоре, а z — число гиперпотоков в каждом ядре. Тогда произведение p = xyz определяет число сущностей, именуемых логическими процессорами системы. Оно определяет полное число независимых контекстов прикладных процессов в системе с общей памятью, исполняющихся параллельно, которые учитывает операционная система.


%\chapter{ Констукторский раздел}
\label{cha:design}
\section{ Разработка алгоритмов}

\subsection{ Классический алгоритм умножения матриц}
В листенге \ref{list:std} представлен стандартный алгоритм умножения матриц.

\begin{lstlisting}[caption={Псевдокод стандартного алгоритма умножения матриц.}, label={list:std}]
пока i < число строк матрицы А:
	пока j < число элементов в строке матрицы B:
        пока k < число строк матрицы B:
		    Результирующая матрица C[i][j] += A[i][k]*B[k][j]
\end{lstlisting}	



\subsection{ Классический алгоритм Винограда умножения матриц}
В листинге \ref{list:win} представлен классический алгоритм умножения Винограда

\begin{lstlisting}[caption={Псевдокод алгоритма умножения матриц Винограда.}, label={list:win}]
d = число строк B // 2
пока i < число строк матрицы А:
	пока j < d:
		rows[i] += A[i][2*j]*A[i][2*j + 1]

пока i < число столбцов B:
	пока j < d:
		colms[i] += B[2*j][i]*B[2*j+1][i]

пока i < число строк A:
	пока j < число cтолбцов B:
        результат[i][j] = -rows[i] - colms[i]
        пока k < d:
            результат[i][j] += (A[i][2*k] + B[2*k+1][j]) * (A[i][2*k+1] + B[2*k][j])

если количество элементов в строке матрицы A нечетная:
    пока i < число строк в A:
        пока j < число элементов в строке B:
            результат[i][j] += A[i][индекс последнего элемента строки A] * B[индекс последнего элемента строки A][j];
\end{lstlisting}	




%\chapter{ Технологический раздел}
\label{cha:impl}

Данный раздел содержит информацию о реализации ПО и листингах програм.

\section{ Требования к програмному ПО}
Программы реализованы в двух вариациях. Первая -- каждая программа реализована поотдельности и обладает своим собственным функционалом -- возможность определять размеры матриц и получать вывод в поток вывода. Вторая -- программа, содержащая в себе функции с реализованными алгоритмами, в которой производится тестирование времени алгоритмов и вывод отправляется в поток вывода информации.

\section{ Средства реализации }
Данные реализации разрабатывались на языке C, использовался компилятор gcc-8.2.1.
Замер времени производится с помощью библиотеки \textit{time.h}. Данная библиотека позволяет производить замер тиков, откуда можно получать время в секундах. 
Программы компилировались с выключенной оптимизацией с флагом -O0

\section{ Листинг}
Алгоритм Винограда классический представлен в листинге \ref{list:winograd}
\begin{lstlisting}[style=CStyle, caption={Winograd algorithm},
                    label={list:winograd}]

void winograd(  int** a, int ra, int ca , 
                int** b, int rb, int cb, 
                int** c, int rc, int cc,
                int* rows, int* columns)
{
    // f1 = 2 + 2 + ra(2 + 6 + 1 + 1 + 1 + 2 + d(2 + 6 + 1 + 2 + 3))=
    //    = ra*d*14 + 13*ra + 2
    int d = ca / 2;

    for(int i = 0; i < ra; i++) {
        rows[i] = rows[i] + a[i][0] * a[i][1];
        for(int j = 1; j < d; j++)
            rows[i] = rows[i] + a[i][2*j] * a[i][2*j+1];
    }

    // f2 = 2 + cb(2 + 6 + 1 + 1 + 1 + 2 + d(2 + 6 + 1 + 2 + 3))
    //    = cb * d * 14 + 13*cb + 2
    for(int i = 0; i < cb; i++) {
        columns[i] = columns[i] + b[0][i]*b[1][i];
        for(int j = 1; j < d; j++)
            columns[i] = columns[i] + b[2*j][i] * b[2*j+1][i];
    }

    // f3 = 2 + ra(2 + 2 + cb(2 + 4 + 2 + 1 + 2 + d(2 + 12 + 1 + 5 + 5))) =
    //    = 25*ra*cb*d + 11*ra*cb + 4*ra + 2
    for(int i = 0; i < ra; i++)
        for(int j = 0; j < cb; j++) {
            c[i][j] = -rows[i] - columns[j];
            for(int k = 0; k < d; k++)
                c[i][j] = c[i][j] + (a[i][2*k] + b[2*k+1][j]) * 
                            (a[i][2*k+1] + b[2*k][j]);
        }

    // f4 = 1 + ( (0) or (ra*cb*14 + 4*ra + 2) )

    if(ca%2)
        for(int i = 0; i < ra; i++)
            for(int j = 0; j < cb; j++)
                c[i][j] = c[i][j] + a[i][ca-1] * b[ca-1][j];

    // fwinogradaBest = 25*ra*cb*d + 11*ra*cb + 14*ra*d + 14*cb*d + 17*ra + 13*cb + 9
    // fwinogradWorst = 25*ra*cb*d + 25*ra*cb + 14*ra*d + 14*cb*d + 21*ra + 13*cb + 11
}
\end{lstlisting}

Оптимизированный алгоритм Винограда представлен в листинге \ref{list:winogradEnh}

\begin{lstlisting}[style=CStyle, caption={Winograd enhanced algorithm},
                    label={list:winogradEnh}]
void winogradEnhanced(  int** a, int ra, int ca , 
                        int** b, int rb, int cb, 
                        int** c, int rc, int cc,
                        int* rows, int* columns)
{
    // f1 = 2 + 2 + ra(2 + 2 + d(2 + 5 + 1 + 1 + 1))=
    //    = 10*ra*d + 4*ra + 4
    int d = ca - 1;
    for(int i = 0; i < ra; i++) {
        for(int j = 0; j < d; j+=2)
            rows[i] += a[i][j] * a[i][j+1];
    }

    // f2 = 2 + cb(2 + 2 + d(2 + 5 + 1 + 1 + 1))=
    //    = 10*cb*d + 4*cb + 4
    for(int i = 0; i < cb; i++) {
        for(int j = 0; j < d; j+=2)
            columns[i] += b[j][i] * b[j+1][i];
    }

    // f3 = 2 + ra(2 + 2 + cb(2 + 4 + 2 + 1 + 2 + d(2 + 10 + 1 + 4 + 1))) =
    //    = 18*ra*cb*d + 11*ra*cb + 4*ra + 2
    for(int i = 0; i < ra; i++)
        for(int j = 0; j < cb; j++) {
            c[i][j] = -rows[i] - columns[j];
            for(int k = 0; k < d; k+=2)
                c[i][j] += (a[i][k] + b[k+1][j]) * 
                            (a[i][k+1] + b[k][j]);
        }

    // f4 = 1 + ( (0) or (ra*cb*8 + 4*ra + 2) )

    if(ca%2)
        for(int i = 0; i < ra; i++)
            for(int j = 0; j < cb; j++)
                c[i][j] += a[i][d] * b[d][j];

    // fwinogradEnhancedBest = 18*ra*cb*d + 10*cb*d + 10*ra*d + 11*ra*cb + 8*ra + 4*cb + 10
    // fwinogradEnhancedWorst = 18*ra*cb*d + 10*cb*d + 10*ra*d + 19*ra*cb + 12*ra + 4*cb + 12
}
\end{lstlisting}

Листинг стандартного перемножения матриц представлен в \ref{list:classicM}

\begin{lstlisting}[style=CStyle, caption={Classic matrix multiplication},
                    label={list:classicM}]
void classic(           int** a, int ra, int ca , 
                        int** b, int rb, int cb, 
                        int** c, int rc, int cc)
{
    for(int i = 0; i < ra; i++)
        for(int j = 0; j < cb; j++)
            for(int k = 0; k < rb; k++)
                c[i][j] += a[i][k] * b[k][j];
    // fstd = 2 + ra(2 + 2 + cb(2 + 2 + rb(1 + 1 + 8 + 1 + 2))) = 13*ra*cb*rb + 4*ra*cb + 4*ra + 2
}
\end{lstlisting}



%\chapter{ Экспериментальный раздел}

В данном разделе будут произведены замеры времени работы алгоритмов Бойера-Мура и КМП. Для исследования скоростных характеристик был использован компьютер на базе процессора Intel Core Intel(R) Core(TM) i5-8250U CPU @ 1.60GHz, содержащий 8 Гб оперативной памяти.

\section{ Тестирование алгоритма}

\begin{table}[]
\resizebox{\textwidth}{!}{%
\begin{tabular}{lcccc}
\multicolumn{1}{c}{\textbf{Строка}} & \textbf{Подстрока} & \textbf{\begin{tabular}[c]{@{}c@{}}Бойера-Мур, \\ мсек\end{tabular}} & \textbf{\begin{tabular}[c]{@{}c@{}}КМП, \\ мсек\end{tabular}} & \textbf{Результат} \\
\begin{tabular}[c]{@{}l@{}}"the quick brow fox \\ jumps over the lazy \\ dog"\end{tabular} & "dog" & 8.678436279296875e-05 & 2.7894973754882812e-05 & 40
\end{tabular}%
}
\end{table}



\backmatter %% Здесь заканчивается нумерованная часть документа и начинаются ссылки и
            %% заключение

%\Conclusion % заключение к отчёту
Изучен алгоритм Кнута-Морриса-Пратта и  Бойера-Мура. Реализованы алгоритмы Кнута-Морриса-Пратта и  Бойера-Мура. Экспериментально подтверждены различия во временной эффективности алгоритмов Кнута-Морриса-Пратта и  Бойера-Мура.


%% % Список литературы при помощи BibTeX
% Юзать так:
%
% pdflatex rpz
% bibtex rpz
% pdflatex rpz

\bibliographystyle{gost780u}
\bibliography{rpz}


%%% Local Variables: 
%%% mode: latex
%%% TeX-master: "rpz"
%%% End: 


%\appendix   % Тут идут приложения

%%\chapter{Картинки}
%\label{cha:appendix1}

%\begin{figure}
%\centering
%\caption{Картинка в приложении. Страшная и ужасная.}
%\end{figure}

%%% Local Variables: 
%%% mode: latex
%%% TeX-master: "rpz"
%%% End: 

%%\chapter{Еще картинки}
%\label{cha:appendix2}

%\begin{figure}
%\centering
%\caption{Еще одна картинка, ничем не лучше предыдущей. Но %надо же как-то заполнить место.}
%\end{figure}

%%% Local Variables: 
%%% mode: latex
%%% TeX-master: "rpz"
%%% End: 


\end{document}

%%% Local Variables:
%%% mode: latex
%%% TeX-master: t
%%% End:
