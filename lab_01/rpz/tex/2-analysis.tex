\chapter{ Аналитический раздел}
\label{cha:analysis}

Перед теоритическим изложением алгоритмов, представленных в работе, требуется ввести понятия \textit{редукционного расстояния} и \textit{метода динамического программирования}. 

\textit{Редукционное расстояние(расстояние Эйнштейна)} -- это минимальное количество  редукционных операций, необходимых для преобразования одной строки в другую. 

Есть следующие редукционные операции:
\begin{itemize}
\item Операции, вес которых - 1:
\begin{itemize}
 \item I - insert(вставка);
 \item D - delete(удаление);
 \item R - replace(замена);
\end{itemize}
\item Операция, вес которой - 0:
\begin{itemize}
 \item M - match(совпадение);
\end{itemize}
\end{itemize}

Так минимальное расстояние между строками minD('увлечение', 'развлечение') = 3, но чтобы найти это, требуется перебрать расстояния с разным выравниваем строк по отношеню друг к другу.\\

\begin{tabu} to 0.8\textwidth { | X[c] | X[c] | X[c] | X[c] | X[c] | X[c] | X[c] | X[c] | X[c] | X[c] | X[c] | }
 \hline
    &  & у & в & л & е & ч & е & н & и & е \\
 \hline
    р & а & з & в & л & е & ч & е & н & и & е  \\
 \hline
    I & I & R & M & M & M & M & M & M & M & M  \\
\hline
\end{tabu}\\

Проблема выравнивания решается рекуррентно через расстояния между подстроками фиксированной длины.

\section{ Описание Алгоритмов}
