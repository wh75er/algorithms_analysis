\chapter{ Аналитический раздел}
\label{cha:analysis}

Перед теоритическим изложением алгоритмов, представленных в работе, требуется ввести понятия \textit{редукционного расстояния} и \textit{метода динамического программирования}. 

\textit{Редукционное расстояние(расстояние Эйнштейна)} -- это минимальное количество  редукционных операций, необходимых для преобразования одной строки в другую. 

Есть следующие редукционные операции:
\begin{itemize}
\item Операции, вес которых - 1:
\begin{itemize}
 \item I - insert(вставка);
 \item D - delete(удаление);
 \item R - replace(замена);
\end{itemize}
\item Операция, вес которой - 0:
\begin{itemize}
 \item M - match(совпадение);
\end{itemize}
\end{itemize}

Так минимальное расстояние между строками minD('увлечение', 'развлечение') = 3, но чтобы найти это, требуется перебрать расстояния с разным выравниваем строк по отношеню друг к другу.\\

\begin{tabu} to 0.8\textwidth { | X[c] | X[c] | X[c] | X[c] | X[c] | X[c] | X[c] | X[c] | X[c] | X[c] | X[c] | }
 \hline
    &  & у & в & л & е & ч & е & н & и & е \\
 \hline
    р & а & з & в & л & е & ч & е & н & и & е  \\
 \hline
    I & I & R & M & M & M & M & M & M & M & M  \\
\hline
\end{tabu}\\

Проблема выравнивания решается рекуррентно через расстояния между подстроками фиксированной длины.

\section{ Описание Алгоритмов}
Первым появившимся алгоритмом был алгоритм Левенштейна, который заложил фундамент в поиске расстояния между строками.

\subsection{ Расстояние Левенштейна}
Расстояние Левенштейна имеет широкую область применения. Алгоритм используется в:
\begin{itemize}
    \item поисковых системях, в базах данных, в автоматическом распознавание текста и речи для исправления ошибок и опечаток в слове;
    \item в утилитах для сравнения файлов(таких как diff);
    \item в биоинформатике для сравнения генов, хромосом и белков;
\end{itemize}

Алгоритм рекуррентно через расстояния между подстроками i и j находит расстояние между строками s1 и s2. Проверяя, какое действие будет наиболее выгодным I(insert), D(delete), M(match) или R(replace): 

\begin{equation}
    D(s1, s2) = D(s1[1..i], s2[1..j]) = 
    min 
    \left (
    \begin{tabular}{c}
        D(s1[1..i], s2[1..j-1]) + 1, \\
        D(s1[1..i-1], s2[1..j]) + 1, \\
        D(s1[1..i-1], s2[1..j-1]) + \Bigg[
        \begin{tabular}{c}
            0, если s1[i] = s2[j],\\
            1, иначе
        \end{tabular}\\
    \end{tabular}
    \right )
    \label{eq:levenst}
\end{equation}
\textit{,где 
$i$ -- длина подстроки строки s1, которая изначально равно длинне s1, 
$j$ -- длина подстроки строки s2, которая изначально равно длинне s2} \\

Таким образом, применив метод динамического програмированая мы разбиваем нашу задачу на небольшие подзадачи, которые достаточно легко можно решить. Данный метод удобно представлять в виде матрицы. \\

\begin{tabu} to 0.8\textwidth { | X[c] | X[c] | X[c] | X[c] | }
 \hline
    & $\lambda$ & М & Г\\
 \hline
    $\lambda$ & 0 & 1 & 2 \\
 \hline
    М & 0 & 0 & 1\\
\hline
\end{tabu}
\footnote{(Представление D('М', 'МГ'))}\\

В виде матрицы редукционные операции можено представить следующим образом:
\begin{itemize}
    \item I(insert) $\rightarrow$;
    \item D(delete) $\downarrow$;
    \item M(match) or R(replace) $\searrow$;
\end{itemize}

У данного алгоритма есть улучшенная версия, которую модифицировал Фредерик Дамерау.

\subsection{ Расстояние Дамерау -- Левенштейна}
Данный алгоритм применятеся также как и обычный в:
\begin{itemize}
    \item поисковых системах;
    \item биоинформатике(сравнение белков линейной структуры);
\end{itemize}

Причиной появления данного алгоритма было огромное количество ошибок ввода -- ввод двух соседних символов не в том порядке. Отсюда появляется новая операция в дополнении к уже имеющимся:
\begin{itemize}
        \item X - exchange(или T - transposition);
\end{itemize}

Отсюда, формула \ref{eq:levenst} переходит в:
\begin{equation}
D(s1[1..i], s2[1..j]) = 
    min 
    \left (
    \begin{tabular}{c}
        D(s1[1..i], s2[1..j-1]) + 1, \\
        D(s1[1..i-1], s2[1..j]) + 1, \\
        D(s1[1..i-1], s2[1..j-1]) + \Bigg[
        \begin{tabular}{c}
            0, если s1[i] = s2[j],\\
            1, иначе
        \end{tabular},\\
        D(s1[1..i-2], s2[1..j-2]) + 1 \\
    \end{tabular} 
    \right )
    \label{eq:damerlevenst}
\end{equation}

Причем последняя операция(X) выполняется, если такая перестановка необхадима и требуется, и это не совпадение.
