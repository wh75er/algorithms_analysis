\Introduction

Целью работы является изучение и применение метода динамического программирования на материале алгоритмов Левенштейна и Дамерау-Левенштейна, а так же реализовать алгоритм Левенштейна в рекурсивном виде. Для достижения поставленной цели необходимо решить следующие задачи:

\begin{itemize}
\item Изучить алгоритмы Левенштейна и Дамера-Левенштейна нахождения расстояния между строками;
\item Применить метод динамического програмирования для матричной реализации указаных алгоритмов;
\item Получить практические навыки реализации указанных алгоритмов: двух алгоритмов в матричной версии и алгоритма Левенштейна, реализованного рекурсивно;
\item Провести сравнительный анализ линейной и рекурсивной реализаций алгоритма Левенштейна по затрачиваемым ресурсам(времени и памяти);
\item Привести эксперементальное подтверждение различий во временной эффективности рекурсивной и нерекурсивной реализаций алгоритма Левенштейна при помощи разработанного ПО на материале замеров процессорного времени выполнения реализации на варьирующихся длинах строк;
\item Описать и обосновать полученные результаты о выполненной работе;
\end{itemize}
