\chapter{ Констукторский раздел}
\label{cha:design}

Ниже представлены схемы алгоритмов -- Дамерау - Левенштейна и двух реализаций Левенштейна(рекурсивный и обычный).

\section{ Разработка алгоритмов}
\subsection{ Расстояние Левенштейна(обычный)}

\begin{tikzpicture} [
    auto,
    decision1/.style = { diamond, draw=black, thick, 
                        text width=5em, text badly centered,
                        inner sep=1pt },
    block/.style    = { rectangle, draw=black, thick, 
                        text width=10em, text centered,
                        minimum height=2em },
    line/.style     = { draw, thick, ->, shorten >=2pt },
    terminator/.style = { ellipse, thick, draw=black, text badly centered },
    startstop/.style = {rectangle, rounded corners, minimum width=3cm, 
                        minimum height=1cm,text centered, draw=black, 
                        fill=red!30},
    io/.style       = { trapezium, trapezium left angle=70, 
                        trapezium right angle=110, minimum width=3cm, 
                        minimum height=1cm, text centered, draw=black, 
                        fill=blue!30},
    process/.style  = { rectangle, minimum width=3cm, minimum height=1cm, 
                        text centered, draw=black, fill=orange!30},
    decision/.style = { diamond, minimum width=3cm, minimum height=1cm, 
                        text centered, draw=black, fill=green!30},
  ]
  % Define nodes in a matrix
  \matrix [column sep=5mm, row sep=10mm] {
                    & \node [startstop] (start) {Начало};            & \\
                    & \node [process] (doa) {$\begin{tabular}{c}
                                            i=len(s), \\ 
                                            j=len(t), \\
                                            matrix[i][j], \\
                                            cost=1
                                            \end{tabular}$};   & \\
  };
  % connect all nodes defined above
  \begin{scope} [every path/.style=line]
    \path (start)        --    (doa);
  \end{scope}
  %
  % legend for subprocedures
  \node (leyend) at (7.5, 5){
    \begin{tabular}{>{\sffamily}l@{: }l}
      \multicolumn{2}{c}{\textbf{subprocedures}} \\
      DoAE & direction of arrival estimation     \\
      UID  & user identification                 \\
      DoAT & DoA tracking                        \\
      BF   & beam forming                        \\
      SF   & spatial filtering
    \end{tabular}
  };
  %
  % legend for input and output variables
  \node (leyend) at (7, 0){
    \begin{tabular}{l@{: }l}
      \multicolumn{2}{c}{\textbf{variables}}              \\
      DoA                       & direction of arrival    \\
      $\mathbf{i}$              & identification sequence \\
      $\mathbf{X},\,\mathbf{x}$ & signal model            \\
      DoA$_{\mathrm{T}}$        & DoAs up to date         \\
      $\hat{x}(t)$              & fitered signal
      \end{tabular}
  };
\end{tikzpicture}
