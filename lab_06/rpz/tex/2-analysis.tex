\chapter{ Аналитический раздел}
\label{cha:analysis}

Описание работы конвейра.

\section { Вычислительные конвейеры}

Конвейер -- способ организации вычислений, который помогает повысить производительность обработки данных. Такая технология используется в контроллерах и процессах.Идея заключается в параллельном выполнении нескольких инструкций процессора. Сложные инструкции процессора представляются в виде последовательности более простых стадий. Вместо выполнения инструкций последовательно (ожидания завершения конца одной инструкции и перехода к следующей), следующая инструкция может выполняться через несколько стадий выполнения первой инструкции. Это позволяет управляющим цепям процессора получать инструкции со скоростью самой медленной стадии обработки, однако при этом намного быстрее, чем при выполнении эксклюзивной полной обработки каждой инструкции от начала до конца. 

\section{ Описание алгоритма}

На вход конвейеру подается массив чисел -- очередь, обрабатываемых чисел. Конвейер делится на три части -- уровень A, уровень B, уровень C:

\textit{Уровень A}: берет число из очереди и находит факториал этого числа, после чего передает его в очередь Уровня B.

Уровень B: Получает число из очереди, после чего домнажает его на 3, передавая в очередь уровня C. 

Уровень C: Получает число из очереди уровня C и проверяет его на делимость на 3. Если число делится на 3, оно заносится в результат.

