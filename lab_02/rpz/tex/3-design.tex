\chapter{ Констукторский раздел}
\label{cha:design}
\section{ Разработка алгоритмов}

\subsection{ Классический алгоритм умножения матриц}

\begin{lstlisting}[caption={Pseudo code of classic matrix multiplication}]
пока i < число строк матрицы А:
	пока j < число элементов в строке матрицы B:
        пока k < число строк матрицы B:
		    Результирующая матрица C[i][j] += A[i][k]*B[k][j]
\end{lstlisting}	



\subsection{ Классический алгоритм Винограда умножения матриц}

\begin{lstlisting}[caption={Pseudo code of classic Winograd algorithm}]
d = число строк B // 2
пока i < число строк матрицы А:
	пока j < d:
		rows[i] += A[i][2*j]*A[i][2*j + 1]

пока i < число столбцов B:
	пока j < d:
		colms[i] += B[2*j][i]*B[2*j+1][i]

пока i < число строк A:
	пока j < число cтолбцов B:
        результат[i][j] = -rows[i] - colms[i]
        пока k < d:
            результат[i][j] += (A[i][2*k] + B[2*k+1][j]) * (A[i][2*k+1] + B[2*k][j])

если количество элементов в строке матрицы A нечетная:
    пока i < число строк в A:
        пока j < число элементов в строке B:
            результат[i][j] += A[i][индекс последнего элемента строки A] * B[индекс последнего элемента строки A][j];
\end{lstlisting}	



