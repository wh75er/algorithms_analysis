\chapter{ Констукторский раздел}
\label{cha:design}
\section{ Разработка алгоритмов}

\subsection{ Классический алгоритм умножения матриц}

\begin{lstlisting}
пока i < число строк матрицы А:
	пока j < число элементов в строке матрицы B:
        пока k < число строк матрицы B:
		    Результирующая матрица C[i][j] += A[i][k]*B[k][j]
\end{lstlisting}	



\subsection{ Классический алгоритм Винограда умножения матриц}

\begin{lstlisting}
b = число строк B // 2
пока i < число строк матрицы А:
	пока j < число строк B // 2:
		rows[i][j] += A[i][2*j]*A[i][2*j + 1]

пока i < число столбцов B:
	пока j < число строк B // 2:
		colms[i] += B[2*j][i]*B[2*j][i]

пока i < число строк A:
	пока j < число cтолбцов B // 2:
		если размерность матрицы четная:
			результат[i][j] = sum(A[i][b-1]*B[b-1][j])
		иначе:
			результат[i][j] = sum((A[i][2 * k] + B[2 * k + 1][j]) * (A[i][2 * k + 1] + B[2 * k][j]) - rows[i] - colms[j]
\end{lstlisting}	



