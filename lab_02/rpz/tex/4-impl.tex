\chapter{ Технологический раздел}
\label{cha:impl}

Данный раздел содержит информацию о реализации ПО и листингах програм.

\section{ Требования к програмному ПО}
Программы реализованы в двух вариациях. Первая -- каждая программа реализована поотдельности и обладает своим собственным функционалом -- возможность определять размеры матриц и получать вывод в поток вывода. Вторая -- программа, содержащая в себе функции с реализованными алгоритмами, в которой производится тестирование времени алгоритмов и вывод отправляется в поток вывода информации.

\section{ Средства реализации }
Данные реализации разрабатывались на языке C, использовался компилятор gcc-8.2.1.
Замер времени производится с помощью библиотеки \textit{time.h}. Данная библиотека позволяет производить замер тиков, откуда можно получать время в секундах. 
Программы компилировались с выключенной оптимизацией с флагом -O0

\section{ Листинг}
Алгоритм Винограда классический представлен в листинге \ref{list:winograd}
\begin{lstlisting}[style=CStyle, caption={Winograd algorithm},
                    label={list:winograd}]

void winograd(  int** a, int ra, int ca , 
                int** b, int rb, int cb, 
                int** c, int rc, int cc,
                int* rows, int* columns)
{
    // f1 = 2 + 2 + ra(2 + 6 + 1 + 1 + 1 + 2 + d(2 + 6 + 1 + 2 + 3))=
    //    = ra*d*14 + 13*ra + 2
    int d = ca / 2;

    for(int i = 0; i < ra; i++) {
        rows[i] = rows[i] + a[i][0] * a[i][1];
        for(int j = 1; j < d; j++)
            rows[i] = rows[i] + a[i][2*j] * a[i][2*j+1];
    }

    // f2 = 2 + cb(2 + 6 + 1 + 1 + 1 + 2 + d(2 + 6 + 1 + 2 + 3))
    //    = cb * d * 14 + 13*cb + 2
    for(int i = 0; i < cb; i++) {
        columns[i] = columns[i] + b[0][i]*b[1][i];
        for(int j = 1; j < d; j++)
            columns[i] = columns[i] + b[2*j][i] * b[2*j+1][i];
    }

    // f3 = 2 + ra(2 + 2 + cb(2 + 4 + 2 + 1 + 2 + d(2 + 12 + 1 + 5 + 5))) =
    //    = 25*ra*cb*d + 11*ra*cb + 4*ra + 2
    for(int i = 0; i < ra; i++)
        for(int j = 0; j < cb; j++) {
            c[i][j] = -rows[i] - columns[j];
            for(int k = 0; k < d; k++)
                c[i][j] = c[i][j] + (a[i][2*k] + b[2*k+1][j]) * 
                            (a[i][2*k+1] + b[2*k][j]);
        }

    // f4 = 1 + ( (0) or (ra*cb*14 + 4*ra + 2) )

    if(ca%2)
        for(int i = 0; i < ra; i++)
            for(int j = 0; j < cb; j++)
                c[i][j] = c[i][j] + a[i][ca-1] * b[ca-1][j];

    // fwinogradaBest = 25*ra*cb*d + 11*ra*cb + 14*ra*d + 14*cb*d + 17*ra + 13*cb + 9
    // fwinogradWorst = 25*ra*cb*d + 25*ra*cb + 14*ra*d + 14*cb*d + 21*ra + 13*cb + 11
}
\end{lstlisting}

Оптимизированный алгоритм Винограда представлен в листинге \ref{list:winogradEnh}

\begin{lstlisting}[style=CStyle, caption={Winograd enhanced algorithm},
                    label={list:winogradEnh}]
void winogradEnhanced(  int** a, int ra, int ca , 
                        int** b, int rb, int cb, 
                        int** c, int rc, int cc,
                        int* rows, int* columns)
{
    // f1 = 2 + 2 + ra(2 + 2 + d(2 + 5 + 1 + 1 + 1))=
    //    = 10*ra*d + 4*ra + 4
    int d = ca - 1;
    for(int i = 0; i < ra; i++) {
        for(int j = 0; j < d; j+=2)
            rows[i] += a[i][j] * a[i][j+1];
    }

    // f2 = 2 + cb(2 + 2 + d(2 + 5 + 1 + 1 + 1))=
    //    = 10*cb*d + 4*cb + 4
    for(int i = 0; i < cb; i++) {
        for(int j = 0; j < d; j+=2)
            columns[i] += b[j][i] * b[j+1][i];
    }

    // f3 = 2 + ra(2 + 2 + cb(2 + 4 + 2 + 1 + 2 + d(2 + 10 + 1 + 4 + 1))) =
    //    = 18*ra*cb*d + 11*ra*cb + 4*ra + 2
    for(int i = 0; i < ra; i++)
        for(int j = 0; j < cb; j++) {
            c[i][j] = -rows[i] - columns[j];
            for(int k = 0; k < d; k+=2)
                c[i][j] += (a[i][k] + b[k+1][j]) * 
                            (a[i][k+1] + b[k][j]);
        }

    // f4 = 1 + ( (0) or (ra*cb*8 + 4*ra + 2) )

    if(ca%2)
        for(int i = 0; i < ra; i++)
            for(int j = 0; j < cb; j++)
                c[i][j] += a[i][d] * b[d][j];

    // fwinogradEnhancedBest = 18*ra*cb*d + 10*cb*d + 10*ra*d + 11*ra*cb + 8*ra + 4*cb + 10
    // fwinogradEnhancedWorst = 18*ra*cb*d + 10*cb*d + 10*ra*d + 19*ra*cb + 12*ra + 4*cb + 12
}
\end{lstlisting}

Листинг стандартного перемножения матриц представлен в \ref{list:classicM}

\begin{lstlisting}[style=CStyle, caption={Classic matrix multiplication},
                    label={list:classicM}]
void classic(           int** a, int ra, int ca , 
                        int** b, int rb, int cb, 
                        int** c, int rc, int cc)
{
    for(int i = 0; i < ra; i++)
        for(int j = 0; j < cb; j++)
            for(int k = 0; k < rb; k++)
                c[i][j] += a[i][k] * b[k][j];
    // fstd = 2 + ra(2 + 2 + cb(2 + 2 + rb(1 + 1 + 8 + 1 + 2))) = 13*ra*cb*rb + 4*ra*cb + 4*ra + 2
}
\end{lstlisting}


