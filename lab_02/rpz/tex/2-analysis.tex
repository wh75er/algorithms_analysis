\chapter{ Аналитический раздел}
\label{cha:analysis}

Умножение матриц -- важная задача в разных областях, которая помогает рассчитать системы линейных уравнение, которые в свою очередь применяются в системах моделирования реального мира, 3D моделировании, в обработке и хранении информации. 

Задача по эффективной обработке матриц является актуальной и востребованной в наши дни. Первым шагом на пути к более быстрому по времени перемножении матриц стал алгоритм Винограда. Он представляет собой перемножение матриц с использованием скалярного произведения соответствующих строки и столбца исходных матриц.

Более подробное описание алгоритма: 

\section{ Алгоритм Винограда}

Пусть даны матрицы A[N*M] и B[M*K], тогда их результирующее произведение можно представить как матрицу C[N*K]. При умножении "в лоб" мы получаем много операций умножения, которые являются очень трудоемкими по сравнению со сложением:

\begin{equation}
    A * B = v_1*w_1 + v_2 * w_2 + v_3 * w_3 + ... + v_m * w_m
    \label{eq:basis}
\end{equation}

Но это выражение можно представить в другом виде, в котором хорошо будут видны элементы, которые возможно рассчитать заранее и использовать несколько раз. Рассмотрим на частном случае:
Пусть даны вектора $V = (v1, v2, v3, v4)$ и $W = (w1, w2, w3, w4)$. Произведение $V*W$ можно найти по формуле \ref{eq:basis}, которая представлена в общем виде, но можно записать так:

\begin{equation}
    V*W = (v_1 + w_2)*(v_2 + w_1) + (v_3 + w_4)*(v_4+w_3) - v_1 * v_2 - v_3 * v_4 - w_1 * w_2 - w_3 * w_4
    \label{eq:win}
\end{equation}

Часть $- v_1 * v_2 - v_3 * v_4 - w_1 * w_2 - w_3 * w_4$ из выражения \ref{eq:win} вычислима заранее, что позволят нам избавиться от лишних трудоемких операций умножения.

\section{ Рассчет трудоемкости алгоритма}

Рассчет трудоемкости алгоритма опирается на модель вычисления, в которой нужно определить свод правил : оценка операций, циклов, стоимость перехода по условиям.
