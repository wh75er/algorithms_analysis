\chapter{ Технологический раздел}
\section{ Требования к програмному обеспечению}
    Данная программа разрабатывалась на языке python с использованием интерпретатора python 3.7.1. Для работы с массивами данных использовалась библиотека NumPy.

\section{ Листинг кода}

Листинг алгоритма муравья представлен в  \ref{list:levelc}

\begin{lstlisting}[language=python, caption={ Реализация уровня C конвейера},
                    label={list:levelc}]

def aco(m, e, d, tmax, alpha, beta, p, q):
    nue = 1 / d
    teta = np.random.sample((m, m))
    Tmin = None
    Lmin = None

    t = 0

    while t < tmax:
        tetak = np.zeros((m, m))

        for k in range(m):
            Tk = [k]
            Lk = 0
            i = k

            while len(Tk) != m:
                J = [r for r in range(m)]
                for c in Tk:
                    J.remove(c)

                P = [0 for alpha in J]

                for j in J:
                    if d[i][j] != 0:
                        buf = sum((teta[i][l] ** alpha) * (nue[i][l] ** beta) for l in J)
                        P[J.index(j)] = (teta[i][j] ** alpha) * (nue[i][j] ** beta) / buf
                    else:
                        P[J.index(j)] = 0

                Pmax = max(P)
                if Pmax == 0:
                    break

                index = P.index(Pmax)
                Tk.append(J[index])
                Lk += d[i][J[index]]
                i = J.pop(index)

            if Lmin is None or (Lk + d[Tk[0]][Tk[-1]]) < L_min:
                Lmin = Lk + d[Tk[0]][Tk[-1]]
                Tmin = Tk

            for g in range(len(Tk) - 1):
                alpha= Tk[g]
                betha = Tk[g + 1]
                tetak[alpha][betha] += q / Lk

        tetae = (e * q / Lmin) if Lmin else 0
        teta = (1 - p) * teta + tetak + tetae
        t += 1

    return Lmin

\end{lstlisting}
