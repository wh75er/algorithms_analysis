\chapter{ Аналитический раздел}
\label{cha:analysis}

Поиск подстроки в строке является распространенной задачей в повседневной жизни. С данной задачей можно столкнуться при поиске какой-либо информации. Все самые известные сайты, хранящие какие-либо данные, будь то пользователи, видео или изображения, статьи, предоставляют поиск по этой информации.

Для решения задачи поиска подстроки в строке используются различные алгоритмы и их модификации. Одни из них: алгоритм Кнута-Морриса-Пратта и алгоритм Бойера-Мура.

\section { Алгоритм Кнута-Морриса-Пратта}

Дана цеочка T и образец P. Требутеся найти все позиции, начиная с которох P входит в T. Построим строку $S=P\#T$, где $\#$ -- любойоо символ, не входящий в алфавит P
и T. Посчитаем на ней значение префикс-фукнции p. Благодаря разделительному символу \#, выполняется $\forall :p[i] \leq |P|$. Заметим, что по определению префикс-функции при $i > |P|$ и $p[i]=|P|$ подстроки длины P, начинающиеся с позиции 0 и $i-|P|+1$, совпадают. Соберем все такие позиции $i-|P|+1$ строки S, вычтем из каждой позиции $|P|+1$, это и будет ответ. Другими словами, если в какой-то позиции i выполняется условие $p[i]=|P|$, то в этой позиции начинается очередное вхождение образца в цепочку.

\section { Алгоритм Бойера-Мура}

На первом этапе строится таблица смещений для искомой строки. далее совмещается начало строки и образа и начинается проверка с последнего символа образа. Если последний символ образа и соответствующий ему символ строки не совпадают, то образ сдвигается оттносительно строки вправо на величину, полученную из таблицы смещений, и снова проводится сравнение начиная с последнего символа образа. в противном случае, при совпадении символов, производится сравнение предпоследнего символа образа и т.д. Если все символы образа совпали с соответствующими символами строки, то нужная подстрока найдена. Если же какой-то (не последний) символ образа не совпадает с с соответствующим символом строки, то образ сдвигается на один символ вправо и снова начинается проверка с последнего символа.

Таблица смещений строится по следующим правилам:
\begin{enumerate}
    \item сдвиг искомого образа должен быть минимальным, таким, чтобы не пропустить вхождение образа в строке;
    \item если данный символ строки встречается в образе, то он сдвигается таким образом, чтобы символ строки совпла с самым правым вхождением этого символа в образе;
    \item если искомая строка не содержит этого символа, то образ сдвигается на длину искомой строки.
\end{enumerate}

Величина смещения для каждого символа искомой строки зависит только от порядка символов в ней самой, поэтому таблицу смещений вычисляют заранее и чаще всего хранят в виде одномерного массива, где каждому символу из искомой строки соответствует смещение относительно  последнего символа.


