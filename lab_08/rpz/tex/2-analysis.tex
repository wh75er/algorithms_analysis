\chapter{ Аналитический раздел}
\label{cha:analysis}
В данном разделе приведено описание алгоритма Муравья.

\section { Постановка задачи комммивояжера}

Пусть $I = {1, .., n}$ -- множество городом, матрица $c_{ij}$ -- попарные расстояния между городами, $i \leq 1, j \leq n$

Требуется найти контур минимальной длины, цикл, проходящий через каждую вершину одоин раз и имеющий минимальный вес.

Данная задача может быть решена полоным перебором всех вариантов. Но с увеличением числа городов количество маршрутов очень быстро возрастает, так как для n городов есть $n!$ возможных марршрутовю. Даже если учесть, что для одного контура есть два возможных направления и искать только одно из них, количество переборово снижается только до $\frac{(n-1)!}{2}$ вариантов.

\section { Муравьиный алгоритм}

Муравьиный алгоритм -- один из эффективных алгоритмов нахождения решений задачи коммивояжера. Алгоритм является жадным эврестическим алгоритмом: алгоритмом, принимающим локально оптимальные решения для каждой итерации.

Алгоритм основывается на поведении муравьев в поиске кратчайшего пути от колонии до источника питания. Моделирование поведения муравья связано с распределением феромонов на пути -- на ребре в данном случае, количество феромона пропорционально длине маршрута. Таким образом, чем короче найденный маршрут, тем больше будет на нем феромонов. Чтобы алгоритм не сводился к единственному варианту, вводится обратная связь -- "испарение" феромонов с пути со временем.

Муравьиный алгоритм использует три основных понятия -- память муравья, видимость и феромонный след.

Память муравья -- список всех вершин, которые он уже посетил. Так же есть список $J_{i, k}$ -- список вершин, которые нужно посетить i-тому муравью, находящемуся в вершине k.

Видимость -- обратная расстоянию велечина, $\eta = \frac{1}{D_{ij}}$, где $D_{ij}$ -- расстояние между вершинами i и j.

Феромонный след -- "опыт" муравьев, коэффициент вероятности того, что муравей захочет перейти из вершины i в j -- $\tau_{ij}$.

Вероятность перехода муравья k в вершину j из вершины i указано в выражении \ref{eq:chance}

\begin{equation}
P_{i, j, k} = \frac{[\tau_{ij}]^\alpha * [\eta_{ij}]^\beta}{\sum_{l \in J} [\tau_{i,l}]^\alpha * [\eta_{j,l}]^\beta}
\label{eq:chance}
\end{equation}

Если $\alpha = 0$, то алгоритм становится "жадным", если $\beta = 0$, то муравьи оказываются "слепы" и опираются только на феромонный след.

В конце каждой итерации высчитывается общая протяженность маршрута и обнуляется память муравьев.
