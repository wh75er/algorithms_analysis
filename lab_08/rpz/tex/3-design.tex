\chapter{ Констукторский раздел}
\label{cha:design}
\section{ Разработка алгоритмов}

Псевдокод алгоритма КМП представлен в листинге \ref{list:kmp}
Псевдокод Таблицы суффиксов-префиксов, использующаяся в данной реализации KMP, представлен в листинге \ref{list:lsp}

Псевдокод алгоритма Бойера-Мура представлен в листинге \ref{list:bm}

\begin{lstlisting}[caption={Псевдокод алгоритма КМП}, label={list:kmp}]
функция КМП:
    на вход:
        строка S
        подстрока W
    на выход:
        p - индекс строки S, с которой начинается W

    инициялизация:
        j = 0 - индекс строки S
        k = 0 - нидекс подстроки W
        T = рассчитать таблицу префиксов для W(массив)

    пока j < length(S)
        пока k>0 и S[j] != W[k]
            k = T[k-1]
        если S[j] == W[k]
            k += 1
            если k == длине строки W
                вернуть j - (k-1)

    вернуть -1
\end{lstlisting}

\begin{lstlisting}[caption={Псевдокод создания таблицы префиксов для подстроки}, label={list:lsp}]
функция lsp:
    на вход:
        подстрока W
    на выход:
        таблица наибольших суффиксов-префиксов 

    инициализация:
        lsp = [0] * длинну подстроки W

    для i от 1 до длины подстроки W
        j = lsp[i-1]
        пока j > 0 и W[i] != W[j]
            j = lsp[j-1]
        если W[i] == W[j]
            j++
        lsp[i] = j

    вернуть lsp
\end{lstlisting}

\begin{lstlisting}[caption={ Псевдокод алгоритма Бойера-Мура}, label={list:bm}]
функция БМ
    на вход:
        строка S
        подстрока W
    на выход:
        p - индекс строки S, с которой начинается W
    
    m = длина подстроки W
    i = m-1
    j = m-1
повторять
    если P[j] = T[i]
        если j=0
            вернуть i
        иначе 
            i = i -1
            j = j -1
    иначе
        i = i + m - Min(j, 1 + last[T[i]])
        j = m -1
пока i > n -1
\end{lstlisting}
