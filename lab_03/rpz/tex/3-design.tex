\chapter{ Констукторский раздел}
\label{cha:design}
\section{ Разработка алгоритмов}

\subsection{ Алгоритм сортировки пузырька}
В листенге \ref{list:bubblePseudo} представлен алгоритм сортировки пузырьком.

\begin{lstlisting}[caption={Pseudocode of bubble sort}, label={list:bubblePseudo}]
цикл для j от min до max
    цикл для i от min до max-1
        если A[i] больше чем A[i+1] то:
            обменять местами A[i] и A[i+1]
\end{lstlisting}	


\subsection{ Алгоритм сортировки слиянием}
В листинге \ref{list:mergePseudo} представлен алгоритм сортировки слиянием.

\begin{lstlisting}[caption={Pseudocode of merge sort}, label={list:mergePseudo}]
функция mergesort( массив А )
    Если n=1 то:
        вернуть А

    l1 = A[0] .. A[n/2]
    l2 = A[n/2+1] .. A[n]

    l1 = mergesort( l1 )
    l2 = mergesort( l2 )

    вернуть merge( l1, l2 )
конец функции

функция merge( массив A, массив B )
    пока в A и в B есть элементы:
        если A[0] > B[0]:
            добавить B[0] в конец C
            удалить B[0] из B
        иначе
            добавить A[0] в конец C
            удалить A[0] из A
    пока в A есть элементы:
        добавить A[0] в конец C
        удалить A[0] из A
    пока в B есть элементы:
        добавить B[0] в конец C
        удалить B[0] из B
    вернуть C
конец функции
\end{lstlisting}	

\subsection{ Алгоритм сортировки простыми вставками}
Листинг алгоритма сортировки простыми вставками представлен в \ref{list:insertionsPseudo}

\begin{lstlisting}[caption={Pseudocode of insertions algorithm}, label={list:insertionsPseudo}]
цикл для i от 0 до n - 1
    j = i - 1
    пока j >= 0 и a[j] > a[j + 1] 
        поменять местами a[j] и a[j + 1]
        j--
\end{lstlisting}
