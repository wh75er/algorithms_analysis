\chapter{ Технологический раздел}
\label{cha:impl}

Данный раздел содержит информацию о реализации ПО и листингах програм.

\section{ Требования к програмному ПО}
Программы реализованы в двух вариациях. Первая -- каждая программа реализована поотдельности и обладает своим собственным функционалом -- возможность определять размеры массива и получать вывод в поток вывода. Вторая -- программа, содержащая в себе функции с реализованными алгоритмами, в которой производится тестирование времени алгоритмов и вывод отправляется в поток вывода информации.

\section{ Средства реализации }
Данные реализации разрабатывались на языке C, использовался компилятор gcc-8.2.1.
Замер времени производится с помощью библиотеки \textit{time.h}. Данная библиотека позволяет производить замер тиков, откуда можно получать время в секундах. 
Программы компилировались с выключенной оптимизацией с флагом -O0

\section{ Листинг}
Алгоритм сортировки пузырьком представлен в листинге \ref{list:bubble}

\begin{lstlisting}[style=CStyle, caption={bubble sort},
                    label={list:bubble}]
void bubble(int* a, int size)
{
    for(int i = 0; i < size; i++)
        for(int j = 0; j < size-j-i; j++)
            if(a[j] > a[j+1])
                swap(&a[j], &a[j+1]);
}
\end{lstlisting}

Алгоритм сортировки слиянием представлен в листинге \ref{list:merge}

\begin{lstlisting}[style=CStyle, caption={merge sort},
                    label={list:merge}]
void merge(int* a, int* l, int leftCount, int* r, int rightCount)
{
    int i = 0, 
        j = 0, 
        k = 0;

    while(i<leftCount && j< rightCount) {
        if(l[i]  < r[j]) a[k++] = l[i++];
        else a[k++] = r[j++];
    }
    while(i < leftCount) a[k++] = l[i++];
    while(j < rightCount) a[k++] = r[j++];
}

void mergeSort(int* a, int n)
{
    int mid, i, *l, *r;
    if(n < 2) return;

    mid = n/2;

    l = (int*)malloc(mid*sizeof(int)); 
    r = (int*)malloc((n - mid)*sizeof(int)); 
	
    for(i = 0; i<mid; i++) l[i] = a[i]; // creating left subarray
    for(i = mid; i<n; i++) r[i-mid] = a[i]; // creating right subarray

    mergeSort(l, mid);
    mergeSort(r, n-mid);
    merge(a, l, mid, r, n-mid);
    free(l);
    free(r);
}
\end{lstlisting}

Листинг алгоритма сортировки простыми вставками представлен в \ref{list:insertions}

\begin{lstlisting}[style=CStyle, caption={Insertions sort},
                    label={list:insertions}]
void insert(int* a, int size)
{
    int j, tmp;
    
    for(int i = 1; i < size; i++) {
        tmp = a[i];
        for (j = i - 1; j >= 0 && a[j] > tmp; j--)
            a[j + 1] = a[j];
        a[j + 1] = tmp;
    }
}
\end{lstlisting}


